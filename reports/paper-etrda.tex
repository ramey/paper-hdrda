\documentclass[11pt]{article}
\usepackage{graphicx, amsmath, amssymb, bm, url, mathtools, natbib, amsthm, setspace}

\pagestyle{plain}
%----------------Page dimensions ----------------
\oddsidemargin 0.0in
\evensidemargin 0.0in
\topmargin -0.75in
\leftmargin 0in
\headheight 0.0in
\headsep 0.5in
%\footheight 0.0in
\footskip 0.5in
\footnotesep 0.0in
\textwidth 6.7in
\textheight 9.5in
%-----------Define Pictures---------------------
\def\picture #1 by #2 (#3){
 \vbox to #2{
   \hrule width #1 height 0pt depth 0pt
   \vfill
   \special{picture #3} % this is the low-level interface
   }
 }
\def\scaledpicture #1 by #2 (#3 scaled #4){{
 \dimen0=#1 \dimen1=#2
 \divide\dimen0 by 1000 \multiply\dimen0 by #4
 \divide\dimen1 by 1000 \multiply\dimen1 by #4
 \picture \dimen0 by \dimen1 (#3 scaled #4)}
 }

\newcommand{\xbar}{\bar{\bm x}}
\newcommand{\tr}{\text{tr}}
\DeclareMathOperator*{\argmin}{arg\,min}
\DeclareMathOperator*{\argmax}{arg\,max}

\title{Another Look at Regularized Discriminant Analysis for High-dimensional Classification}

\author{John A. Ramey, Caleb K. Stein, and Dean M. Young}

\begin{document}

\newtheorem*{thm}{Theorem}
\newtheorem*{cor}{Corollary}

\bibpunct{(}{)}{;}{a}{,}{,}

\doublespacing

\maketitle



\begin{abstract}
\cite{Friedman:1989tm} proposed the popular regularized discriminant analysis (\emph{RDA}) classifier for the small-sample, high-dimensional setting by employing a regularized covariance matrix estimator, which employs a ridge-like shrinkage factor to ensure positive definiteness. We demonstrate that the \emph{RDA} classifier, in its original form, is inadequate for the small-sample, high-dimensional setting. We redefine the \emph{RDA} classifier to increase its computational runtime and its interpretation. We show that this new definition includes several well-known regularized classifiers as special cases. We also consider several properties of the newly defined \emph{RDA} classifier. Alternatively, \cite{Srivastava:2007ww} propose an empirical-Bayesian estimator that employs a similar shrinkage factor using the average of the nonzero eigenvalues.  Because the eigenvalues are often highly skewed, this average is often excessively large, so that the near-zero eigenvalues are substantially overcompensated. A comparable eigenvalue adjustment method from \cite{Thomaz:2006ef} often yields a shrinkage factor that is approximately equal to the average of the eigenvalues. By writing each shrinkage method as a matrix function, we generalize the shrinkage method of \cite{Thomaz:2006ef} to allow a powerful estimation technique that does not over inflate the near-zero eigenvalues. By incorporating our proposed covariance-matrix regularization with the RDA classifier, we overcome the numerical incalculability issues of the \emph{RDA} classifier. Moreover, we show that our proposed classifier is often superior to the regularized classifiers mentioned above in terms of classification accuracy using three simulated and three small-sample, high-dimensional microarray data sets. Additionally, we provide implementations of the competing classifiers in the {\tt regdiscrim} R package, available on CRAN.
\end{abstract}

\section{Introduction}

Microarrays have enabled researchers to measure gene-expression levels of thousands of genes in a single experiment to identify a subset of genes and their expression patterns that either characterize a particular cell state or facilitate an accurate prediction of future cell state.  We focus on the classification of tissues and other biological samples into one of $K$ disjoint, predefined cancer types, where our goal is to determine accurately the class membership of a new tissue sample based on gene-expression microarray data. However, microarray data typically consist of a large number of gene expressions $p$ relative to the sample size $n$. Moreover, standard benchmark microarray data sets consist of between $p = 10,000$ and $55,000$ probes and fewer than $n = 100$ observations, in which case standard supervised-learning methods exhibit poor classification performance, due to the \emph{curse of dimensionality} \citep{Bellman:1961tn}. Furthermore, well-known classifiers, including linear discriminant analysis \emph{(LDA)} and quadratic discriminant analysis \emph{(QDA)}, are incalculable when $p > n$ because the sample covariance matrices employed in both classifiers are singular. As a result, many researchers, including \cite*{Merchante:2012vk}, \cite{Witten:2011kc}, \cite*{Pang:2009ik}, \cite*{Huang:2010ju}, and \cite*{Clemmensen:2011kr}, have restricted the \emph{LDA} and \emph{QDA} models to reduce number of covariance parameters that must be estimated to ensure that the resulting classifiers are calculable.  For instance, \cite{Dudoit:2002ev} have proposed a modified \emph{LDA} classifier where the class covariance matrices are assumed to be diagonal.

Alternatively, several researchers, including \cite*{Zhang:2010va}, \cite*{Guo:2007te}, \cite{Srivastava:2007ud}, and \cite{Mkhadri:1995jp}, have proposed regularization methods to improve the estimation of the population covariance matrices before applying the \emph{LDA} and \emph{QDA} classifiers. Covariance-matrix regularization methods often are based on the well-known ridge-regression approach from \cite{Hoerl:1970cd} and adjust the eigenvalues of the sample covariance matrices to ensure that the estimator is positive definite. This shrinkage stabilizes eigenvalues that are near zero to ensure that the inverse of the the sample covariance matrix is well-posed. For an excellent overview of the regularization problem in discriminant analysis, see \cite{Murphy:2012uq} and \cite{Mkhadri:1997gy}.

Here, we consider specifically the  popular regularized discriminant analysis (\emph{RDA}) classifier from \cite{Friedman:1989tm} that incorporates a convex combination of the sample covariance matrices from the \emph{LDA} and \emph{QDA} classifiers. Because the resulting biased covariance-matrix estimator often remains singular in the small-sample, high-dimensional setting, \cite{Friedman:1989tm} applies covariance-matrix shrinkage, where the resulting covariance-matrix estimator is shrunken towards a scaled identity matrix. \cite{Friedman:1989tm} have shown that the \emph{RDA} classifier can attain improved classification accuracy, compared to the \emph{LDA} and \emph{QDA} classifiers when $n > p$.

In this paper we present a generalization of the \emph{RDA} classifier in terms of matrix functions. We show that several common covariance-matrix regularization methods can be specified in terms of a matrix function including the \emph{RDA} classifier as well as methods from \cite{Srivastava:2007ww} and \cite*{Thomaz:2006ef} and several sparse diagonal classifiers, including the diagonal \emph{LDA} (\emph{DLDA}) and \emph{QDA} (\emph{DQDA}) classifiers from \cite{Dudoit:2002ev}, and some of its variants including those proposed by \cite{Pang:2009ik}.  Our generalization elucidates some properties of these covariance-regularization methods and the relationships among them. Moreover, our generalization allows us to view the covariance-matrix regularization methods in terms of eigenvalue thresholds, which further allows us to identify situations where each method is ineffective and ineffective. From a perspective of eigenvalue thresholds, we propose two new classifiers that are often superior to the \emph{RDA} classifier in terms of classification performance because both methods employ a more robust eigenvalue thresholding to regularize the convex combination of covariance matrix estimators employed in the \emph{RDA} classifier. Our two proposed methods also have an advantage over the \emph{RDA} classifier because both methods employ dimension reduction by reducing the number of basis functions included in the \emph{RDA} classifier, thereby reducing the variance of the resulting classifiers.

In our first classifier we generalize the eigenvalue replacement employed in the \emph{NLDA} classifier and apply a soft threshold to the eigenvalues of the \emph{RDA} covariance-matrix estimator. We refer to our first classifier as the \emph{RDA} with soft eigenvalue-thresholding (\emph{Soft-ETRDA}) classifier.  As we show, the eigenvalue replacement employed in the \emph{NLDA} classifier is often approximately equal to the shrinkage applied in the \emph{RDA} classifier for small-sample, high-dimensional data. Furthermore, a special case of the \emph{Soft-ETRDA} classifier is the \emph{NLDA} classifier as well as the \emph{RDA} classifier.  We remark that our soft eigenvalue threshold is similar to and partially motivated by the shrunken centroids classifier from \cite*{Tibshirani:2002ht}. 	In our second proposed classifier, we apply a hard threshold to the eigenvalues of the \emph{RDA} covariance-matrix estimator, so that the number of covariance parameters incorporated into the \emph{RDA} classifier is reduced substantially. We refer to this classifier as the \emph{RDA} with hard eigenvalue-thresholding (\emph{Hard-ETRDA}) classifier. We remark that the inverse of \emph{Hard-ETRDA} covariance-matrix estimator includes as a special case, the Moore-Penrose pseudo inverse of the \emph{RDA} covariance-matrix estimator.

Using three small-sample, high-dimensional microarray data sets, we demonstrate that our two proposed classifiers often yield superior classification accuracy to the \emph{RDA}, \emph{MDEB}, and \emph{NLDA} classifiers. We also include in our comparison the alternatively-defined \emph{RDA} classifier from \cite*{Hastie:2008dt}, the regularized \emph{LDA} classifier from \cite{Guo:2007te}, a recently proposed variant of diagonal discriminant analysis from \cite*{Tong:2012hw}, and the penalized \emph{LDA} classifier from \cite{Witten:2011kc}. Additionally, we examine the classification accuracy of the classifiers considered using \textbf{three} simulation configurations. Furthermore, we show that our proposed eigenvalue-threshold methods are more numerically stable than the other considered regularization methods using the simulated configurations.

We have organized the remainder of the paper as follows. In Section 2 we review the \emph{RDA}, the \emph{MDEB}, and \emph{NLDA} classifiers and define each in terms of matrix functions. In Section 3 we generalize the eigenvalue adjustment employed in the \emph{NLDA} classifier and define our proposed \emph{Soft-ETRDA} and \emph{Hard-ETRDA} classifiers. In Section 4 we describe the \textbf{three} simulation configurations and three small-sample, high-dimensional microarray data sets and then demonstrate that our two proposed classifiers are superior to the competing classifiers in terms of classification accuracy. We then conclude with a brief discussion in Section 6.

\section{Regularized Discriminant Analysis}

We\footnote{Define the majority of notation here, including $z^+$, $z_+$, and $I[z]$.} wish to assign correctly an unlabeled $p$-dimensional observation vector $\bm x$ to one of $K$ unique, known classes (or populations) by constructing a classifier from $n$ training observations that can accurately predict the class membership of $\bm x$. Let $\bm x_i = (x_{i1}, x_{i2}, \ldots, x_{ip}) \in \mathbb{R}_{p \times 1}$ be the $i$th observation $(i = 1, \ldots, n)$ with true, unique membership $y_i \in \mathcal{K} = \{\omega_1, \ldots, \omega_K\}$, where $\mathbb{R}_{m \times n}$ denotes the matrix space of all $m \times n$ matrices over the real field $\mathbb{R}$. We assume that $(\bm x_i, y_i)$ is a realization from a mixture distribution $p(\bm x) = \sum_{k=1}^K p(\bm x | \omega_k) p(\omega_k)$, where $p(\bm x | \omega_k)$ is the probability density function (PDF) of the $k$th class, $p(\omega_k)$ is prior probability of class membership of the $k$th class.

The \emph{QDA} classifier is the optimal Bayesian decision rule with respect to a $0-1$ loss function when $p(\bm x | \omega_k)$ is the PDF of the multivariate normal distribution with known mean vector  $\bm\mu_k \in \mathbb{R}_{p \times 1}$ and known covariance matrix $\bm\Sigma_k \in \mathbb{R}_{p \times p}^{>}$, $k = 1, 2, \ldots, K$, where $\mathbb{R}_{p \times p}^{>}$ denotes the cone of real $p \times p$ positive-definite matrices. Because the parameters are typically unknown, we estimate the unknown parameters $\bm \mu_k$ and $\bm\Sigma_k$ with their maximum likelihood estimators (MLEs) and substitute the MLEs into the \emph{QDA} classifier. Assuming, for simplicity, that the prior probabilities of class membership $p(\omega_k)$ are equal for $k = 1, \ldots, K$, we assign an unlabeled observation $\bm x$ to class $\omega_k$ using the sample \emph{QDA} classifier:
\begin{align}
	D_{QDA}(\bm x) = \argmin_{k}  (\bm x - \xbar_k)'\widehat{\bm\Sigma}_k^{-1}(\bm x - \xbar_k)  + \ln |\widehat{\bm\Sigma}_k|, \label{eq:qda}
\end{align}
where $\xbar_k$ and $\widehat{\bm\Sigma}_k$ are the MLEs for $\bm \mu_k$ and $\bm \Sigma_k$, respectively. If we assume further that the covariance matrix parameters are equal for each class (i.e., $\bm\Sigma_k = \bm\Sigma$ for all $k$), then \eqref{eq:qda} reduces to Fisher's linear discriminant analysis (LDA) classifier,
\begin{align}
	D_{LDA}(\bm x) = \argmin_{k}  (\bm x - \xbar_k)'\widehat{\bm\Sigma}^{-1}(\bm x - \xbar_k), \label{eq:lda}
\end{align}
where $\widehat{\bm\Sigma}$ is the pooled sample covariance matrix
\begin{align}
	\widehat{\bm\Sigma} = \frac{1}{n} \sum_{k=1}^K n_k \widehat{\bm\Sigma}_k. \label{eq:pooled-cov}
\end{align}

\subsection{Original Definition of Regularized Discriminant Analysis}

\cite{Friedman:1989tm} has proposed the \emph{RDA} classifier by incorporating a weighted average of the sample covariance matrix $\widehat{\bm \Sigma}_k$ for class $\omega_k$ and the pooled sample covariance matrix $\widehat{\bm\Sigma}$ to estimate the covariance matrix for class $\omega_k$ with
\begin{align}
  \widehat{\bm\Sigma}_k(\lambda) = n_k^{-1}(\lambda) \bm S_k(\lambda),\label{eq:sig-lambda}
\end{align}
where $\lambda \in [0, 1]$, $n_k(\lambda) = (1 - \lambda) n_k + \lambda n$, $\bm S_k = n_k \widehat{\bm\Sigma}_k$, $\bm S = \sum_{k=1}^K \bm S_k$, and $\bm S_k(\lambda) = (1 - \lambda) \bm S_k + \lambda \bm S$. We can interpret \eqref{eq:sig-lambda} as a covariance matrix estimator for class $\omega_k$ that borrows from $\widehat{\bm\Sigma}$ in \eqref{eq:pooled-cov} to better estimate $\bm \Sigma_k$. If $\lambda = 0$, \eqref{eq:sig-lambda} is equal to $\widehat{\bm\Sigma}_k$ used in the \emph{QDA} classifier in \eqref{eq:qda}. Alternatively, if $\lambda = 1$, \eqref{eq:sig-lambda} is equal to $\widehat{\bm\Sigma}$ used in the \emph{LDA} classifier in \eqref{eq:lda}, in which case we assume implicitly that $\bm\Sigma_k = \bm \Sigma$ for $k = 1, \ldots, K$.

To further improve the estimation of $\bm \Sigma_k$ and to stabilize the inverse of \eqref{eq:sig-lambda}, \cite{Friedman:1989tm} has proposed the biased \emph{RDA} covariance-matrix estimator
\begin{align}
	\widehat{\bm\Sigma}_k(\lambda, \gamma) = (1 - \gamma) \widehat{\bm\Sigma}_k(\lambda) + \gamma \frac{\tr\{\widehat{\bm\Sigma}_k(\lambda)\}}{p} \bm I_p,\label{eq:sig-rda}
\end{align}
for class $\omega_k$, where $\gamma \in [0, 1]$ is a regularization parameter that controls the shrinkage of \eqref{eq:sig-rda}  towards the $p$-dimensional identity matrix $\bm I_p$, which is weighted by the the average of the eigenvalues of \eqref{eq:sig-lambda}. Thus, the \emph{pooling} parameter $\lambda$ controls the amount that we borrow from $\widehat{\bm\Sigma}$ to estimate $\bm \Sigma_k$, and the \emph{shrinkage} parameter $\gamma$ determines the amount of shrinkage applied.

Substituting \eqref{eq:sig-rda} into \eqref{eq:qda}, we have the \emph{RDA} classifier, where we assign an unlabeled observation $\bm x$ to class $\omega_k$:
\begin{align}
	D_{RDA}(\bm x) = \argmin_{k}  (\bm x - \xbar_k)'\widehat{\bm\Sigma}_k(\lambda, \gamma)^{-1}(\bm x - \xbar_k)  + \ln |\widehat{\bm\Sigma}_k(\lambda, \gamma)|. \label{eq:rda}
\end{align}

\subsection{An Alternative Approach to Regularized Discriminant Analysis}

Here, we present a more straightforward, alternative parameterization of the \emph{RDA} classifier. We first define explicitly
\begin{align}
  \widehat{\bm\Sigma}_k(\lambda) = (1 - \lambda) \widehat{\bm\Sigma}_k + \lambda \widehat{\bm\Sigma}.\label{eq:sig-lambda-alternative}
\end{align}
as a convex combination of the MLEs of $\bm \Sigma_k$ and $\bm \Sigma$, where $\lambda \in [0, 1]$ is the \emph{pooling} parameter. Thus, we no longer have need for the convex combination of $n_k$ and $N$ as defined by \cite{Friedman:1989tm}. Before examining the shrinkage applied to \eqref{eq:sig-lambda-alternative}, we examine some of its properties and write \eqref{eq:sig-lambda-alternative} as
\begin{align}
	\widehat{\bm\Sigma}_k(\lambda) &= \widehat{\bm\Sigma}_k + \lambda (\widehat{\bm\Sigma} - \widehat{\bm\Sigma}_k) \nonumber\\
	&= \widehat{\bm\Sigma}_k + \lambda \left(\sum_{k' = 1}^K \frac{n_k}{N} \widehat{\bm\Sigma}_{k'}  - \widehat{\bm\Sigma}_k \right) \nonumber\\
	&= \left( \frac{1 - \lambda}{n_k} + \frac{\lambda}{N} \right) \widehat{\bm\Sigma}_k +  \frac{\lambda}{N} \sum_{\substack{k' = 1\\k' \ne k}} n_{k'} \widehat{\bm\Sigma}_{k'}.\label{eq:sig-lambda-alternative2}
\end{align}
From \eqref{eq:sig-lambda-alternative2}, we see the weight $\lambda$ places on the covariance matrices from the remaining $K - 1$ classes. The interesting part here is that $\lambda$ controls the contribution of each of the $N$ observations to estimating $\Sigma_k$. Effectively, this states that for $0 < \lambda \le 1$, we use $N$ observations to estimate $\bm \Sigma_k$ rather than $n_k$ as done in the \emph{QDA} classifier. As we will discuss in-depth later, the pooling operation is advantageous in increasing the rank of each $\widehat{\bm\Sigma}_k(\lambda)$ from $n_k - 1$ to $N - 1$ for $0 < \lambda \le 1$. Moreover, suppose that we have centered each observation $\bm x_i$, $i = 1, \ldots, N$, by its class sample mean. Then, we write \eqref{eq:sig-lambda-alternative2} as
\begin{align}
	\widehat{\bm\Sigma}_k(\lambda) &= \left( \frac{1 - \lambda}{n_k} + \frac{\lambda}{N} \right)\sum_{i=1}^N I(y_i = k) \bm x_i \bm x_i' +  \frac{\lambda}{N} \sum_{i=1}^N I(y_i \ne k) \bm x_i \bm x_i' \nonumber \\
	&= \sum_{i=1}^N c_{ik}(\lambda) \bm x_i \bm x_i',\label{eq:sig-lambda-alternative3}
\intertext{where}
	c_{ik}(\lambda) &= \begin{cases}
		(1 - \lambda)n_k^{-1} + \lambda N^{-1}, & y_i = k,\\
		\lambda N^{-1}, & y_i \ne k.\\
	\end{cases}\nonumber
\end{align}
Therefore, after centering each observation by its corresponding class sample mean, we have that $\widehat{\bm\Sigma}_k(\lambda)$ in \eqref{eq:sig-lambda-alternative3} incorporates each centered observation into the estimation of $\bm \Sigma_k$. Notice that if $\lambda  = 0$, then the observations from the remaining $K - 1$ classes do not contribute to the estimation of $\bm \Sigma_k$, corresponding to $\widehat{\bm \Sigma}_k$. Furthermore, if $\lambda = 1$, the weights in \eqref{eq:sig-lambda-alternative3} reduce to $1/N$, corresponding to $\widehat{\bm\Sigma}$.

\subsection{Covariance Matrix Regularization}

Although \eqref{eq:qda} is in a standard form, we prefer to express the \emph{QDA} classifier explicitly in terms of the eigenvalues and eigenvectors of $\widehat{\bm \Sigma}_k$. Let $e_{jk}(\lambda)$ denote the $j$th largest eigenvalue of $\widehat{\bm\Sigma}_k$ and $\bm v_{jk}(\lambda)$ be the corresponding eigenvector such that $e_{1k}(\lambda) \ge e_{2k}(\lambda) \ge \ldots \ge e_{pk}(\lambda) \ge 0$. Equivalently, we write \eqref{eq:qda} as
\begin{align}
  D(\bm x) = \argmin_{k} \sum_{j = 1}^p \frac{[\bm v_{jk}(\lambda)' (\bm x - \bm \xbar_k)]^2}{e_{jk}(\lambda)} + \sum_{j=1}^p \ln \{e_{jk}(\lambda)\}. \label{eq:rda-eigen}
\end{align}
The smallest eigenvalues and the directions associated with their eigenvectors highly influence \eqref{eq:rda-eigen}. In fact, the eigenvalues of $\widehat{\bm \Sigma}_k(\lambda)$ are well-known to be biased if $p \ge n_k$ such that the smallest eigenvalues are underestimated \citep{Seber:2004uh}. This bias increases as $p$ increases relative to $n_k$ \citep{TODONeedCitation}. Moreover, if $p > n_k$, then rank$(\widehat{\bm \Sigma}_k(\lambda)) \le n_k$, which implies that the last $p - n_k$ eigenvalues of $\widehat{\bm \Sigma}_k(\lambda)$ are 0, which implies that \eqref{eq:rda-eigen} is incalculable. Thus, although more feature information is available to discriminate among the $K$ classes as $p$ increases, classification accuracy decreases unless one obtains enough training-sample observations to reliably estimate the increased number of parameters.

Several regularization methods, such as the methods considered by \citet*{Guo:2007te}, \cite{Mkhadri:1995jp}, and \citet*{Xu:2009fl}, have been proposed in the literature to stabilize the eigenvalues of $\widehat{\bm \Sigma}_k$ used in \eqref{eq:qda}. These methods often can be written concisely as a matrix function, which we define here using the notation of \cite{Izenman:2008gm}. Let $\bm A \in \mathbb{R}_{p \times p}$ be symmetric, and let $\phi:\mathbb{R} \rightarrow \mathbb{R}$. Then
\begin{align}
	\phi(\bm A) = \sum_{j = 1}^p \phi(e_j) \bm v_j \bm v_j'\label{eq:spectral-decomp}
\end{align}
is a matrix function defined explicitly in terms of the eigenvalues of $\bm A$. For this reason, $\phi$ is often called a spectral transform \citep{Ch15SSLBook} or a transfer function \citep{Ye:2009gd, Chapelle2002SSLChapter15Reference} . Special case of \eqref{eq:spectral-decomp} include the spectral decomposition of $\bm A$ when $\phi(e_j) = e_j$, the matrix square-root of $\bm A$ when $\phi(e_j) = e_j^{1/2}$, and, if it exists, the inverse of $\bm A$ when $\phi(e_j) = e_j^{-1}$ \citep{Harville:2008wja}.

Viewing regularized covariance-matrix estimators as matrix functions, we can write a more general form of the \emph{QDA} classifier as
\begin{align}
  D(\bm x) = \argmin_{k} \sum_{j = 1}^p \phi(e_{jk}(\lambda))[\bm v_{jk}(\lambda)' (\bm x - \bm \mu_k)]^2 + \sum_{j=1}^p \ln \{ \phi(e_{jk}(\lambda)) \}. \label{eq:generalized-qda}
\end{align}
The\footnote{In \eqref{eq:generalized-qda} we have a slight issue in the formulation of the log-determinant in the second sum. The eigenvalue function $\phi$ in the first sum should be the inverse of the function in the second sum. Furthermore, we need to define the log-determinant so that the last $p - q$ terms are ignored when we reduce the dimension to $q$.} generalized \emph{QDA} classifier in \eqref{eq:generalized-qda} includes several common regularized classifiers. Notice that if  $\phi(e_{jk}(\lambda)) = e_{jk}(\lambda)$, $k = 1, \ldots, K$, \eqref{eq:generalized-qda} reduces to the \emph{QDA} classifier in \eqref{eq:rda-eigen}. Furthermore, if $\phi(e_{jk}(\lambda)) = e_{j}$, $k = 1, \ldots, K$, \eqref{eq:generalized-qda} reduces to the \emph{LDA} classifier in \eqref{eq:lda}, and the second summand in \eqref{eq:generalized-qda} is constant across all classes and can be omitted. Additionally, one common approach is to ignore the zero eigenvalues when $p > n$ by employing the Moore-Penrose pseudoinverse $\widehat{\bm \Sigma}_k^{+}$, which employs the matrix function $\phi(e_j) = e_j^{+}$, where $z^{+} = 1/z$ if $z > 0$ and $0$ otherwise. As noted by \cite{Hoyle:2011vt}, \cite{Raudys:1998dd}, and \cite{Ramey:2011ji}, the pseudoinverse is often inadvisable because the resulting classifier often yields a sharp decrease in classification accuracy if $p \approx n$.

Another common form of the covariance-matrix regularization applies a shrinkage factor $\gamma_k > 0$, so that
\begin{align}
	\phi(e_{jk}(\lambda)) = (e_{jk}(\lambda) + \gamma_k)^{-1}, \label{eq:ridge-estimator}
\end{align}
is a ridge-like estimator, similar to a method employed in ridge regression. The estimator in \eqref{eq:ridge-estimator} effectively \emph{shrinks} the sample covariance matrix $\widehat{\bm\Sigma}_k(\lambda)$ towards the $p$-dimensional identity matrix $\bm I_p$, thereby increasing the eigenvalues of $\widehat{\bm\Sigma}_k(\lambda)$ by $\gamma_k$ so that they are positive, resulting in $\widehat{\bm \Sigma}_k(\lambda) + \gamma_k \bm I_p$ being positive definite.

We emphasize the regularized estimator in \eqref{eq:ridge-estimator} because the \emph{RDA} and \emph{MDEB} classifiers can be written in this form as we show below. Also, we present the \emph{NLDA} and \emph{ETRDA} classifiers in terms of matrix functions using \eqref{eq:generalized-qda}. For additional covariance-matrix regularization methods, see \cite{Ye:2009gd} and \cite{Ramey:2011ji}.

\section{Properties of the Redefined RDA Classifier}

\subsection{Computational Advantages}

Thus far, our alternative parameterization simplifies the definition of the \emph{RDA} classifier and has an improved interpretation in terms of the contribution of each observation. Here, we demonstrate that our parameterization yields significant computational advantages over the originally proposed \emph{RDA} classifier from \cite{Friedman:1989tm} by writing \eqref{eq:sig-lambda-alternative3} as
\begin{align*}
	\widehat{\bm\Sigma}_k(\lambda) &= \bm X_k(\lambda)' \bm X_k(\lambda),
\end{align*}
where $\bm X_k(\lambda) = [\sqrt{c_{1k}(\lambda)} \bm x_1', \ldots,  \sqrt{c_{Nk}(\lambda)} \bm x_N']'$. Now, we can apply the Fast SVD to compute the eigenvalue decomposition of $\widehat{\bm\Sigma}_k(\lambda)$ efficiently. By writing the SVD of $\bm X_k(\lambda) = \bm U_k \bm D_k \bm V_k'$, we have the spectral decomposition $\widehat{\bm\Sigma}_k(\lambda) = \bm X_k(\lambda)' \bm X_k(\lambda) = \bm V_k \bm D_k^2 \bm V_k'$, where $\bm U_k \in \mathbb{R}_{N \times N}$ is orthogonal, $\bm D_k \in \mathbb{R}_{N \times N}$ is a diagonal matrix consisting of the singular values of $\bm X_k(\lambda)$, and $\bm V_k \in \mathbb{R}_{p \times N}$ is orthogonal. Notice that $\bm X_k(\lambda)\bm X_k(\lambda)' = \bm U_k \bm D_k^2 \bm U_k'$, so that the eigenvalue decomposition of $\bm X_k(\lambda)\bm X_k(\lambda)'$ yields $\bm U_k$ and $\bm D_k$. This is much faster when $p \gg N$ because $\bm X_k(\lambda)\bm X_k(\lambda)' \in \mathbb{R}_{N \times N}$. Also, notice that the singular values of $\bm D_k$ are nonzero, so that $\bm D_k^{-1}$ exists. After obtaining $\bm U_k$ and $\bm D_k$, we compute the matrix consisting of the first $N$ eigenvectors of $\widehat{\bm\Sigma}_k(\lambda)$ by simply computing\footnote{TODO: Update this paragraph to theorem/proof.}
\begin{align*}
	\bm V_k = \bm X_k(\lambda)' \bm U_k \bm D_k^{-1}.
\end{align*}

\subsection{Covariance-Matrix Regularization}

We are now ready to discuss the shrinkage employed in the \emph{RDA} covariance-matrix estimator. As we are interested primarily in the case that $p \gg N$, rank$\{\widehat{\bm\Sigma}_k(\lambda)\} = q_k < p$, which implies that $p - q_k$ of the eigenvalues of $\widehat{\bm\Sigma}_k(\lambda)$ are equal to zero. Thus, for $\gamma > 0$, shrinking $\widehat{\bm\Sigma}_k(\lambda)$ towards $\bm I_p$ as in \eqref{eq:sig-rda} increases the rank of $\widehat{\bm\Sigma}_k(\lambda)$ from $q_k$ to $p$. However, \cite{Ye:2009gd} and \cite{YeOtherPaperCitedInTheFirst} have shown that shrinking the zero eigenvalues does not improve the classification performance of the \emph{RDA} classifier. Consequently, we instead shrink only the nonzero eigenvalues of $\widehat{\bm\Sigma}_k(\lambda)$, which has two additional benefits. First, by considering only the nonzero eigenvalues of $\widehat{\bm\Sigma}_k(\lambda)$, we effectively ignore the eigenvectors corresponding to the zero eigenvalues, which reduces significantly the number of parameters that we incorporate into our \emph{RDA} classifier because $p \gg q_k$ if $p \gg N$. Furthermore, the size of the matrix $\bm V_k$ is $p \times N$ as opposed to a $p \times p$ matrix in the original definition of the \emph{RDA} classifier. For significantly large $p$, the reduced size of $\bm V_k$ expedites calculation of the \emph{RDA} classifier and reduces its memory usage.\footnote{TODO: Provide an example for a realistic $p$ and $N$.}

Specifically, we shrink only the nonzero eigenvalues of $\widehat{\bm \Sigma}_k(\lambda)$ by utilizing the estimator

\begin{align}
	\widehat{\bm\Sigma}_k(\lambda, \gamma) = (1 - \gamma) \widehat{\bm\Sigma}_k(\lambda) + \gamma \frac{\tr\{\widehat{\bm\Sigma}_k(\lambda)\}}{p} (\bm I_{q_k} \oplus \bm 0_{p - q_k}),\label{eq:sig-rda-alternative}
\end{align}
where $\bm A \oplus \bm B$ denote the direct sum of $\bm A \in \mathbb{R}_{r \times r}$ and $\bm B \in \mathbb{R}_{s \times s}$ \citep[Chapter 1]{Lutkepohl:1996uz}. Therefore, substituting \eqref{eq:sig-rda-alternative} into \eqref{eq:rda}, we have our alternative parameterization of the \emph{RDA} classifier.


\section{Regularized Discriminant Analysis with Eigenvalue Thresholding}

\subsection{The Hard-ETRDA Classifier}


Although the shrinkage applied in \eqref{eq:cov-nlda} is ineffective, the form in \eqref{eq:cov-nlda} allows us to consider a more general eigenvalue thresholding approach to resolve the numerical issues of the \emph{RDA} classifier discussed below. We are interested in a general shrinkage value $\Delta$ instead of the average of the eigenvalues $\bar{e}_k$. We propose to shrink the pooled covariance matrix estimator in \eqref{eq:sig-lambda} with a more general eigenvalue threshold, motivated by the \emph{NLDA} classifier. That is, for $\Delta_0, \Delta > 0$, we define the matrix function
\begin{align}
	\phi(e_{jk}) = (\Delta_0 + e_{jk} \cdot I\{e_{jk} > \Delta\} )^{+},\label{eq:cov-hard-etrda}
\end{align}
where $e_{jk}$ are the eigenvalues of $\widehat{\bm \Sigma}_k(\lambda)$ in \eqref{eq:sig-lambda} and $\bm v_{jk}$ are the corresponding eigenvectors. Substituting \eqref{eq:cov-soft-etrda} into \eqref{eq:generalized-qda}, we arrive at the \emph{ETRDA} classifier.

For $\Delta_0 = \Delta = \gamma$, the shrinkage employed in \eqref{eq:cov-hard-etrda} is equal to \eqref{eq:ridge-estimator}. Hence, \eqref{eq:cov-hard-etrda} includes the regularization methods of \cite{Friedman:1989tm} and \cite{Srivastava:2007ww} as special cases. To see this, notice that if $\gamma$ is the average of the eigenvalues, \eqref{eq:cov-hard-etrda} is equal to \eqref{eq:rda-eigenvalue-function}. Furthermore, if $\gamma$ is the average of the nonzero eigenvalues \eqref{eq:cov-hard-etrda} is equal to \eqref{eq:cov-mdeb}. Notice also that if $\Delta_0 = \Delta = 0$, then \eqref{eq:cov-hard-etrda} reduces to the Moore-Penrose pseudoinverse \citep{Harville:2008wja}.


\subsection{The Soft-ETRDA Classifier}

As an alternative eigenvalue threshold, we consider a soft threshold. Although the shrinkage applied in \eqref{eq:cov-nlda} is ineffective, the form in \eqref{eq:cov-nlda} allows us to consider a more general eigenvalue thresholding approach to resolve the numerical issues of the \emph{RDA} classifier discussed below. We are interested in a general shrinkage value $\Delta$ instead of the average of the eigenvalues $\bar{e}_k$. We propose to shrink the pooled covariance matrix estimator in \eqref{eq:sig-lambda} with a more general eigenvalue threshold, motivated by the \emph{NLDA} classifier. That is, for $\Delta > 0$, we define the matrix function
\begin{align}
	\phi(e_{jk}) = (\Delta_0 + \{e_{jk} - \Delta\}_+)^{+},\label{eq:cov-soft-etrda}
\end{align}
where $e_{jk}$ are the eigenvalues of $\widehat{\bm \Sigma}_k(\lambda)$ in \eqref{eq:sig-lambda} and $\bm v_{jk}$ are the corresponding eigenvectors. Substituting \eqref{eq:cov-soft-etrda} into \eqref{eq:generalized-qda}, we arrive at the \emph{ETRDA} classifier.

For small values of $\gamma$ such that $\Delta_0 = \Delta = \gamma$, the shrinkage employed in \eqref{eq:cov-soft-etrda} is approximately equal to \eqref{eq:ridge-estimator} because the near-zero are adjusted by $\gamma$, while the larger eigenvalues are hardly adjusted by $\gamma$. Because of this approximation, the regularization methods of \cite{Friedman:1989tm} and \cite{Srivastava:2007ww} are approximately special cases of the \eqref{eq:cov-soft-etrda}. To see this, notice that if $\gamma$ is the average of the eigenvalues, \eqref{eq:cov-soft-etrda} is equal to \eqref{eq:cov-nlda}, which is approximately equal to \eqref{eq:rda-eigenvalue-function}. Furthermore, if $\gamma$ is the average of the nonzero eigenvalues \eqref{eq:cov-soft-etrda} is approximately equal to \eqref{eq:cov-mdeb}.

Notice that \eqref{eq:cov-soft-etrda} has a form similar to the soft thresholding employed in the nearest shrunken centroids classifier from \cite{Tibshirani:2002ht} in the case that the matrix of eigenvectors of $\widehat{\bm \Sigma}_k$ is the identity matrix and $\Delta_0 = 0$. Our choice of $\Delta_0 = 0$ is also advantageous because it can reduce the number of dimensions included in the \emph{Soft-ETRDA} classifier, thereby decreasing the classification reducing. Noting the similarity of \eqref{eq:cov-soft-etrda} to the soft threshold of \cite{Tibshirani:2002ht}, we follow \cite{Tibshirani:2002ht} and select $\Delta$ by cross-validation, resulting in a superior classifier to the \emph{RDA} classifier proposed by \cite{Friedman:1989tm}.

For small values of $\gamma$, the eigenvalue thresholding employed in the \emph{Soft-ETRDA} and \emph{Hard-ETRDA} classifiers are approximately equal. Both methods are effective regularization methods in stabilizing the smallest eigenvalues. The soft eigenvalue threshold employed in the \emph{Soft-ETRDA} classifier has the distinct advantage that the largest eigenvalues are reduced, which smooths the distribution of sample eigenvalues, yielding an effective classifier that is often superior to the proposed sparse and regularized classifiers considered.

\section{Special Cases}

\subsection{RDA}

\cite{Friedman:1989tm} also uses a regularization parameter $\gamma$ to shrink the
eigenvalues of \eqref{eq:sig-lambda} towards the $p$-dimensional identity matrix $\bm I_p$
in order to stabilize the inverse of \eqref{eq:sig-lambda}, resulting in a biased
covariance-matrix estimator with matrix function
\begin{align}
	\phi(e_{jk}(\lambda)) = (\{1 - \gamma\} e_{jk}(\lambda) + \gamma \bar{e}_k)^{-1},\label{eq:rda-eigenvalue-function}
\end{align}
for class $\omega_k$, where $\gamma \in [0, 1]$, $e_{jk}(\lambda)$ be the $j$th largest eigenvalue of $\widehat{\bm\Sigma}_k(\lambda)$ in \eqref{eq:sig-lambda}, $\bm v_{jk}(\lambda)$ be the corresponding eigenvector, and $\bar{e}_k = p^{-1} \sum_{j=1}^p e_{jk}(\lambda)$ is the average of the eigenvalues of \eqref{eq:sig-lambda}.

\subsection{Minimum Distance Empirical Bayes}

\cite{Srivastava:2007ww} derive an empirical-Bayes covariance-matrix estimator under the assumption that the data are independent and identically distributed multivariate normal observations. \cite{Srivastava:2007ww} then substitute the estimator into the \eqref{eq:lda} to attain the minimum distance empirical Bayes (\emph{MDEB}) classifier. To derive the empirical-Bayes covariance-matrix estimator, \cite{Srivastava:2007ww} first assume that $\bm \Sigma_k^{-1}$ follows a Wishart distribution \emph{a priori} with mean $\lambda^{-1} \bm I_p$, $\lambda > 0$, and degrees of freedom, $l \ge p$. The resulting posterior estimator for $\bm \Sigma_k^{-1}$ is $\frac{n_k + l}{n_k}(n_k^{-1}\lambda \bm I_p + \widehat{\bm \Sigma}_k)^{-1}$. Using an empirical Bayes argument to estimate $\lambda$ for $p > n_k$, \cite{Srivastava:2007ww} propose an empirical-Bayes covariance-matrix estimator with matrix function
\begin{align}
	\phi(e_{jk}) = \left(e_{jk} + n_k^{-1} \sum_{j = 1}^p e_{jk}\right)^{-1},\label{eq:cov-mdeb}
\end{align}
where the shrinkage factor in \eqref{eq:cov-mdeb} is approximately equal to the average of the nonzero eigenvalues of $\widehat{\bm \Sigma}_k$. Substituting \eqref{eq:cov-mdeb} into \eqref{eq:generalized-qda}, we have the \emph{MDEB} classifier. We note that \cite{Srivastava:2007ww} set assumed that $\bm \Sigma_k = \bm \Sigma$ so that $\phi(e_{jk}) = \phi(e_k)$, $k = 1, \ldots, K$.

\cite{Srivastava:2007ww} claim that the \emph{MDEB} classifier is best regularization method they have encountered in the literature. When $p \le n$, the shrinkage factors of the \emph{RDA} and \emph{MDEB} classifiers are equal, but for $p > n$ this shrinkage can differ substantially. In fact, \cite{Ramey:2011ji} show empirically that the \emph{MDEB} classifier can yield superior classification accuracy to the \emph{RDA} classifier.


\subsection{New LDA}
Whereas shrinkage methods often have the form given in \eqref{eq:ridge-estimator}, the covariance matrix estimator from \cite{Thomaz:2006ef} stabilizes the inverse by replacing the smallest eigenvalues of \eqref{eq:pooled-cov} with the average of the eigenvalues of \eqref{eq:pooled-cov}. Let $e_{jk}$ denote the $j$th largest eigenvalue of $\widehat{\bm \Sigma}_k$, and let $\bar{e}_k$ be the average of the eigenvalues of \eqref{eq:pooled-cov}. Then, the inverse of the covariance-matrix estimator has the matrix function
\begin{align}
	\phi(e_{jk}) = (\bar{e}_k + \{e_j - \bar{e}_k\}_+)^{-1},\label{eq:cov-nlda}
\end{align}
with $z_+ = z$ if $z > 0$ and 0 otherwise. \cite*{Xu:2009fl} and \cite{Ramey:2011ji} demonstrate that the substituting \eqref{eq:cov-nlda} into \eqref{eq:generalized-qda} yields classification accuracies comparable to the \emph{RDA} and \emph{MDEB} classifiers for small-sample, high-dimensional microarray data sets. Similar to the \emph{MDEB} classifier, \cite{Thomaz:2006ef} assume that $\bm \Sigma_k = \bm \Sigma$ so that $\phi(e_{jk}) = \phi(e_k)$, $k = 1, \ldots, K$.  


\section{Monte Carlo Simulations}

\begin{itemize}
	\item TODO: Design an experiment similar to Friedman's initial simulations, where the the majority of the variances are nearly 0. Fix the sample size. Consider the error rate as a function of $p$. This should cause the RDA regularization to be ineffective because the average eigenvalue is near 0.
	
	\item TODO: Design an experiment where the average eigenvalue goes to 0, but the average nonzero eigenvalue blows up for a fixed sample size. Consider the error rate as a function of $p$. This should cause the RDA and MDEB regularizations to be ineffective. However, I am banking on the ETRDA methods doing well.

\end{itemize}

\section{High-Dimensional Microarray Data Sets}

In this section, we describe three high-dimensional microarray data sets and compare our proposed \emph{ETRDA} classifier with two recently proposed classifiers for small-sample, high-dimensional data: the penalized \emph{LDA} (\emph{PLDA}) classifier from \cite{Witten:2011kc}  and the shrinkage-mean-based DLDA (\emph{SmDLDA}) classifier from \cite{Tong:2012hw}. We calculated 10-fold cross-validation error rates \citep{Hastie:2008dt} to evaluate the performance of each classifier considered. Because the \emph{PLDA} classifier has a built-in variable selection method, for fair comparison we compared each of the classifiers using the variables selected from each cross-validation fold.

\subsection{\cite{Burczynski:2006ik} Data Set}

\cite{Burczynski:2006ik} acquired the peripheral blood mononuclear cells (PBMC) through hybridization to microarrays from 127 individuals resulting in 22,283 sequences. Of the 127 individuals, 42 were healthy, 59 had Crohn's disease (CD), and 26 had ulcerative colitis (UC). The goal of \cite{Burczynski:2006ik} was to improve accuracy in the discrimination of Inflammatory Bowel Disease (IBD) using the PBMC-based gene expression signature of a patient.

\subsection{\cite{Nakayama:2007fl} Data Set}

\cite{Nakayama:2007fl} acquired the gene expression through an oligonucleotide microarray from 105 samples of 10 types of soft tissue tumors. This included 16 samples of synovial sarcoma (SS), 19 samples of myxoid/round cell liposarcoma (MLS), 3 samples of lipoma, 3 samples of well-differentiated liposarcoma (WDLS), 15 samples of dedifferentiated liposarcoma (DDLS), 15 samples of myxofibrosarcoma (MFS), 6 samples of leiomyosarcoma (LMS), 3 samples of malignant nerve sheathe tumor (MPNST), 4 samples of fibrosarcoma (FS), and 21 samples of malignant fibrous histiocytoma (MFH). \cite{Nakayama:2007fl} determined from their data that these 10 types fell into 4 broader groups: (1) SS; (2) MLS; (3) Lipoma, WDLS, and part of DDLS; (4) Spindle cell and pleomorophic sarcomas including DDLS, MFS, LMS, MPNST, FS, and MFH. Following \cite{Witten:2011kc}, we restrict our analysis to the five tumor types having at least 15 samples observed.

\subsection{\cite{Singh:2002fh} Data Set}

\cite{Singh:2002fh} have examined 235 radical prostatectomy specimens from surgery patients between 1995 and 1997. The authors used oligonucleotide microarrays containing probes for approximately 12,600 genes and expressed sequence tags. They have reported that 102 of the radical prostatectomy specimens are of high quality: 52 prostate tumour samples and 50 non-tumour prostate samples.

\subsection{Classification Results}

TODO\footnote{TODO: Add results discussion and table here.}


\section{Discussion}

It is interesting to note that our approach is similar to an aside made by Friedman (1989) in the case that the RDA covariance matrix estimator is singular. Friedman (1989) recommended that the zero eigenvalues be replaced with a small value such as 0.001. However, in Theorem 1, we saw that even small adjustments can be inappropriate because they result in ineffective shrinkage.
	

\bibliographystyle{plainnat}
\bibliography{rda}


\end{document} 
